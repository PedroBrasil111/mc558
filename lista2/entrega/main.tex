\documentclass[
    12pt,
    a4paper,
    brazil,
    english
]{article}
\usepackage{algorithm}
\usepackage{algpseudocode}
\usepackage{titlesec}
\usepackage{graphicx} % For including graphics (e.g., logos)
\usepackage{float} % To place floats (e.g., algorithms) precisely
\usepackage{ragged2e} % For text justification
\usepackage{caption} % To control caption formatting
\usepackage{amsmath} % For better text formatting in math mode

% Document metadata
\title{Lista Avaliativa 2}
\author{Pedro Brasil Barroso - RA 260637}
\date{\today} % Automatically inserts today's date

\captionsetup[algorithm]{labelformat=empty} % Removes numbering from algorithm captions
\renewcommand{\thealgorithm}{} % Removes the numbering but keeps Algoritmo
\setlength{\parindent}{1.5em} % Adjust the 1.5em to control the size of the indentation


\begin{document}

% Custom title page
\begin{titlepage}
    \centering

    % Insert a logo if you want
    %\includegraphics[width=0.2\textwidth]{logo.png} % Replace with your logo's path
    \vspace*{6cm}

    {\LARGE \textbf{MC558 - Lista Avaliativa 2}}
    
    \vspace{5.5cm}
    {\Large Pedro Brasil Barroso - RA 260637}

    \vfill

    {\Large Universidade Estadual de Campinas} \\ % Replace with your university name
    {\Large Instituto de Computação} \\

    \vspace{1cm}
\end{titlepage}

\textbf{\largeÁrvore Geradora Mínima}
\vspace{0.5cm}

- \textbf{Problema escolhido:} \textit{Acacias}, 8 pontos.

\vspace{0.25cm}

- \textbf{Definição do problema:}
\begin{itemize}
    \item \textbf{Entrada:} Um inteiro $N$, que representa o número de vértices, numerados de $1$ a $N$,
    seguido de $N - 1$ linhas. Cada linha $i$ consiste em um inteiro não-negativo $k$, seguido de $k$
    pares de inteiros $j$ e $c_{ij}$, tal que existe uma aresta não direcionada de $i$ para $j$ com peso $c_{ij}$.
    \item \textbf{Saída:} Uma linha com dois inteiros, tais que o primeiro representa o número de componentes conexas no grafo e o segundo, a soma dos custos das árvores mínimas de cada componente.
    \item \textbf{Solução:} O problema pode ser resolvido diretamente com pequenas alterações ao algoritmo de Kruskal visto em aula:
\end{itemize}

% Algorithm Section
\begin{algorithm}
    \caption{\textbf{Algoritmo:} AGM-\textsc{Kruskal}-\textsc{Acacias}($G$, w)}
    \begin{algorithmic}[1]
        \State $A \gets \emptyset$
        \State $componentes \gets |V[G]|$
        \State $custo \gets 0$
        \State \textbf{para cada} $u \in V[G]$
            \State \hspace{1em} \textnormal{\textsc{Make}-\textsc{Set}}($u$)
        \State ordene as arestas em ordem não decrescente de peso
        \State \textbf{para cada} $(u, v) \in E[G]$ na ordem obtida
            \State \hspace{1em} \textbf{se} \textsc{Find-Set}($u$) $\neq$ \textsc{Find-Set}($v$)
            \State \hspace{2em} $A \gets A \cup \{(u, v)\}$
            \State \hspace{2em} \textsc{Union}($u$, $v$)
            \State \hspace{2em} $custo \gets custo + \text{w}(u, v)$
            \State \hspace{2em} $componentes \gets componentes - 1$
            \State \textbf{devolva} $custo, componentes, A$
        \end{algorithmic}
    \end{algorithm}
    \begin{itemize}
        \item \textbf{Complexidade:} Como visto em aula, o algoritmo de Kruskal para árvores geradoras mínimas apresenta complexidade $O(E \log E)$, onde $E$ é o número de arestas do grafo. \\
        Como as operações adicionadas (instanciação e soma nas variáveis "componentes" e "custo") são realizadas em O(1), \textsc{AGM-Kruskal-Acacias} possui complexidade $O(E \log E)$.
        \item \textbf{Corretude:} \\
        Note que, como $G$ pode ser desconexo, \textsc{AGM-Kruskal-Acacias} retorna uma floresta de árvores geradoras mínimas. \\
        Para provar a corretude, é necessário utilizar a seguinte invariante:
        \begin{quote}
            \textit{No início de cada iteração do laço \textbf{para} da linha} 7 \textit{: (1) $pesos$ é a soma dos pesos das arestas em $A$ e (2) $componentes$ é o número de componentes conexas em $G_A = (V, A)$.}
        \end{quote}
        \textbf{Base:} No início do algoritmo, $A = \emptyset$. Inicialmente, $pesos = 0$ e $componentes = |V[G]|$; portanto, a invariante é trivialmente verdadeira. \\\\
        \textbf{Hipótese:} Suponha que a invariante é verdadeira no \mbox{início~da~iteração~$i$}. \\\\
        \textbf{Passo:} Na iteração $i + 1$: \\
    Caso 1: A aresta $(u, v)$ é adicionada a $A$.
    \begin{itemize}
        \item (1) Então $u$ e $v$ são unidos em uma mesma componente conexa, reduzindo o número total de componentes em 1. Na linha 12, $componentes$ é decrementada em 1.
        \item (2) Além disso, o peso da aresta $(u, v)$ é somado \mbox{a $pesos$ na linha 11}.
    \end{itemize}
    Caso 2: A aresta $(u, v)$ não é adicionada a $A$.
    \begin{itemize}
        \item (1) Neste caso, $u$ e $v$ já pertencem à mesma componente conexa, e $componentes$ não é alterada.
        \item (2) O peso da aresta $(u, v)$ não é somado a $pesos$.
    \end{itemize}
    Portanto, ao fim da iteracão $i + 1$, a invariante é mantida. \\\\
    \textbf{Conclusão:} A invariante é mantida em todas as iterações; logo, ao final do algoritmo, $componentes$ é o número de componentes conexas em $G_A$ e $custo$ é a soma dos pesos das arestas em $A$.
\end{itemize}
 
% Caminhos minimos  
\textbf{\large Caminhos mínimos}
\vspace{0.5cm}

- \textbf{Problema escolhido:} \textit{Trip to BH}, 8 pontos.

\vspace{0.25cm}

- \textbf{Definição do problema:}
\begin{itemize}
    \item \textbf{Entrada:} Uma linha contendo dois inteiros N e M, indicando, respectivamente, o número de 
    vértices (numerados de $1$ a $N$) e de arestas. Em seguida, M linhas, cada uma contendo quatro inteiros $A$ $B$ $T$ $R$, indicando que 
    existe uma aresta de $A$ para $B$ com tipo $T$ (0 ou 1) e peso $R$.
    \item \textbf{Saída:} A menor distância entre os vértices $1$ e $N$ dentre os caminhos que utilizam apenas arestas de um mesmo tipo.
    \item \textbf{Solução:} O problema pode ser resolvido diretamente com uma pequena alteração ao algoritmo de Dijkstra visto em aula: basta adicionar um parâmetro $t$ a \textsc{Dijkstra} que indica o tipo da aresta que será considerada na iteração.
    Dessa forma, uma aresta só deverá ser considerada se seu tipo for igual a $t$. Então, basta executar o algoritmo duas vezes, uma para cada tipo de aresta ($t = 0$ e $t = 1$), e comparar os resultados.
    \item \textbf{Complexidade:} Como visto em aula, o algoritmo de Dijkstra para caminhos mínimos apresenta complexidade $O((V + E) \log E)$. No enunciado do exercício, 
    é explicitado que $M \leq 2(N^2 - N)$, então $|V| = N$ e $|E| = O(N^2)$. \\
    Note que a verificação de qual tipo de aresta será considerada é realizada em O(1), pois basta comparar o tipo da aresta antes de adicioná-la à fila de prioridade. \\
    A complexidade total de executar o algoritmo duas vezes, uma para $t = 0$ e outra para $t = 1$, é, portanto, $O(N^2 \log N)$.
    \item \textbf{Corretude:} Sejam $G_0 = (V, E_0)$ e $G_1 = (V, E_1)$, onde $E_t = \{e \in E\ |\ e.T=t\}$ ($e.T$ indica o tipo de $e$).
    Então executar \textsc{Dijkstra} duas vezes, uma para $G_0$ e outra para $G_1$, resulta em caminhos mínimos que contêm apenas arestas de um mesmo tipo. \\
    A alteração sugerida ao algoritmo de Dijkstra garante que, em cada iteração, apenas arestas do tipo $t$ sejam consideradas. Isso equivale a processar apenas o subgrafo $G_t$, provando a corretude dessa solução.\\
\end{itemize}

% Fluxo

\textbf{\large Fluxo em redes}
\vspace{0.5cm}

- \textbf{Problema escolhido:} \textit{My T-Shirt Suits Me}, 6 pontos.

\vspace{0.25cm}

- \textbf{Definição do problema:}
\begin{itemize}
    \item \textbf{Entrada:} \\
    Uma linha contendo dois inteiros $N$ e $M$, $N \ge M$, indicando, respectivamente, o número de 
    camisetas e de voluntários. Em seguida, M linhas, cada uma contendo dois tamanhos de camiseta (dentre \textit{XXL, XL, L, M, S} ou \textit{XS}) que podem ser usados por cada voluntário. N sempre é um múltiplo de 6.\\
    \item \textbf{Saída:} \textit{YES}, se for possível distribuir todas as camisetas de forma que cada voluntário receba uma camiseta de um tamanho que possa usar, ou \textit{NO}, caso contrário.
    \item \textbf{Solução:} \\
    Seja $G = (V, E)$ um grafo que possui um vértice para cada camiseta e um vértice para cada voluntário, e $C, P \subset V$ conjuntos que contêm os vértices associados às camisetas e aos voluntários, respectivamente.
    Adicione uma aresta de capacidade 1 indo de $c \in C$ para $v \in P$ se $v$ puder vestir $c$.
    Então $|V| = 6 + M$ e $|E| = 2M$, pois cada voluntário pode vestir dois tamanhos diferentes, e G é um grafo bipartido com bipartição $(C, P)$. \\
    Em seguida, crie e conecte um vértice $s$ a cada vértice $c \in C$ por uma aresta de capacidade $N/6$, e conecte cada vértice $v \in P$ a um novo vértice $t$ com uma aresta de capacidade 1. 
    Agora, $|V| = 6 + M + 2 = M + 8$ e $|E| = 2M + M + 6 = 3M + 6$\\
    Nessa configuração, se \textit{f} for um fluxo máximo e \mbox{$|f| = M = c(\{s\} \cup C \cup P, \{t\})$}, então é possível distribuir todas as camisetas.
    É possível encontrar $f$ ao executar o algoritmo \textsc{Edmonds-Karp} para a rede $(G, c, s, t)$, onde $c$ armazena as capacidades das arestas, e então retornar \textit{YES} se $|f| = M$, e \textit{NO} caso contrário. \\
    \item \textbf{Complexidade:} 
    \begin{itemize}
        \item A leitura pode ser feita em $O(M)$, pois são lidos $M$ pares de tamanhos de camiseta;
        \item A instanciação do grafo $G$ e a adição das arestas de capacidade 1 podem ser feitas com complexidade $O(M)$, uma vez que $|V| = O(M)$ e $|E| = O(M)$;
        \item A implementação de \textsc{Edmonds-Karp} possui complexidade $O(VE^2)$.
        \item Para descobrir o valor do fluxo máximo, basta somar o fluxo das 6 arestas que saem de $s$, o que pode ser feito em $O(1)$, tal como a checagem para determinar se $|f| = M$ (no algoritmo submetido ao Beecrowd, foi feita uma alteração a \textsc{Edmonds-Karp} para que isso fosse realizado durante sua execução).
    \end{itemize}
    Portanto, com $|V| = M + 8$ e $|E| = 3M + 6$, a complexidade total será:
    $$O(M) + O(M) + O((M + 8)\cdot(3M+6)^2) + O(1) = O(M \cdot M^2) = O(M^3)$$ \\
    \item \textbf{Corretude:} \\
    Sejam $G = (V, E)$, $C$, $P$, $s$ e $t$ definidos como anteriormente.
    Então cada vértice $c \in C$ possui uma aresta de capacidade $N/6$ incidente; portanto, a capacidade total das arestas que saem de $s$ é \mbox{$N/6 \cdot 6 = N$}, representando as N camisetas. Como há arestas de peso 1 indo das camisetas para os voluntários nos quais os tamanhos servem, um fluxo nesse grafo representa uma possível distribuição de camisetas. \\
    Além disso, as únicas arestas incidentes em $t$ são as $M$ arestas de peso 1 que representam os $M$ voluntários. Portanto, \mbox{$c(V - \{t\}, \{t\}) = M$}, logo, sendo $f'$ um fluxo qualquer em G, $|f'| = f(V - \{t\}, \{t\}) \le M$. \\
    Dessa forma, há duas possibilidades para um fluxo máximo $f$ em $G$:
    \begin{enumerate}
        \item $|f| = M$. Então \mbox{$f(v, t) = 1 \ \forall v \in P$}, ou seja, todos os voluntários pertencem a um caminho de $s$ a $t$ em $f$. Nesse caso, cada voluntário receberá uma camiseta que pode vestir.
        \item $|f| < M$. Então \mbox{$(\exists v \in P) \ f(v, t) = 0$}, i.e., ao menos um voluntário não receberá uma camiseta.
    \end{enumerate}
    Portanto, se $|f| = M$, a resposta é \textit{YES}; caso contrário, a resposta é \textit{NO}.

\end{itemize}


\end{document}
