\documentclass[
    12pt,
    a4paper,
    brazil,
    english
]{article}
\usepackage{algorithm}
\usepackage{algpseudocode}
\usepackage{titlesec}
\usepackage{amssymb}
\usepackage{graphicx} % For including graphics (e.g., logos)
\usepackage{float} % To place floats (e.g., algorithms) precisely
\usepackage{ragged2e} % For text justification
\usepackage{caption} % To control caption formatting
\usepackage{amsmath} % For better text formatting in math mode

% Document metadata
\title{Lista Avaliativa 3}
\author{Pedro Brasil Barroso - RA 260637}
\date{\today} % Automatically inserts today's date

%\captionsetup[algorithm]{labelformat=empty} % Removes numbering from algorithm captions
\renewcommand{\thealgorithm}{} % Removes the numbering but keeps Algoritmo
\setlength{\parindent}{1.5em} % Adjust the 1.5em to control the size of the indentation


\begin{document}

% Custom title page
\begin{titlepage}
    \centering

    % Insert a logo if you want
    %\includegraphics[width=0.2\textwidth]{logo.png} % Replace with your logo's path
    \vspace*{6cm}

    {\LARGE \textbf{MC558 - Lista Avaliativa 3}}
    
    \vspace{5.5cm}
    {\Large Pedro Brasil Barroso - RA 260637}

    \vfill

    {\Large Universidade Estadual de Campinas} \\ % Replace with your university name
    {\Large Instituto de Computação} \\

    \vspace{1cm}
\end{titlepage}

\textbf{Problema selecionado:} 7 - \textit{Fábrica de sorvetes}

\textbf{(a) Programa linear explicando as variáveis, função objetivo e restrições.}

\textbf{Variáveis:}
\begin{itemize}
    \item $x_c$: quantidade de sorvete de chocolate (em litros) a ser produzida.
    \item $x_b$: quantidade de sorvete de baunilha (em litros) a ser produzida.
    \item $x_m$: quantidade de sorvete de morango (em litros) a ser produzida.
\end{itemize}

\textbf{Programa linear:}
\begin{equation}
    \text{max } 12x_c + 10x_b + 11x_m
\end{equation}
\begin{align}
    \text{s.a:} \nonumber \\
    0.5x_c + 0.4x_b + 0.45x_m &\leq 500 \\
    0.1x_c + 0.15x_b + 0.12x_m &\leq 80 \\
    10x_c + 8x_b + 9x_m &\leq 2000 \\
    x_c &\geq 80 \\
    x_b &\geq 60 \\
    x_m &\geq 50 \\
    x_i &\geq 0, \text{ para } i \in \{c, b, m\} \\
    x_c, x_b, x_m &\in \mathbb{Q}
\end{align}


\end{document}
